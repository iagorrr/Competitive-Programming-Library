\section{Number Theory}

\subsection{Fundamental theorem of arithmetic}

Every integer greater than 1 can be represented uniquely as a product of prime numbers, up to the order of the factors.

\begin{equation}
    n = p_1^{\alpha_1} p_2^{\alpha_2} ... p_k^{\alpha_k}
\end{equation}

\subsubsection{LCM and GCD}

\begin{equation}
\begin{aligned}
    & a = p_1^{\alpha_1} p_2^{\alpha_2} ... p_k^{\alpha_k}  \\
    & b = p_1^{\beta_1} p_2^{\beta_2} ... p_k^{\beta_k} \\
    & (a,b) = p_1^{\min{{\alpha_1, \beta_1}}} p_2^{\min{{\alpha_2, \beta_2}}} ... p_k^{\min{{\alpha_k, \beta_k}}} \\
    & [a,b] = p_1^{\max{{\alpha_1, \beta_1}}} p_2^{\max{{\alpha_2, \beta_2}}} ... p_k^{\max{{\alpha_k, \beta_k}}}
\end{aligned}
\end{equation}


\subsection{Fermat's Theorems}

Let P be a prime number and $a$ an integer, then:
$$a^p \equiv a \quad (\text{mod } p)$$
$$a^{p-1} \equiv 1 \quad (\text{mod } p)$$

\textbf{Lemma:} Let $p$ be a prime number and $a$ and $b$ integers, then: 
$$(a+b)^{p} \equiv a^{p} + b^{p} \quad (\text{mod } p)$$

\textbf{Lemma:} Let $p$ be a prime number and $a$ an integer. The inverse of $a$ modulo $p$ is $a^{p-2}$:

$$a^{-1} \equiv a^{p-2} \quad (\text{mod } p)$$

\subsection{Goldbach's Conjecture}

"Todo número par, maior que 2, pode se ser escrito como a soma de dois primos. ". Válido pra todo número até $4 \cdot 10^{18}$, mas não tem prova.

Para os números ímpares, sempre pode ser escrito como soma de 3 números primos, tira 3, e pega os dois que formam o número par que restar, porém caso o número primo - 2 seja primo, daí ele consegue ser escrito como 2 também.

\subsection{Taking modulo at the exponent}

Se $a$ e $m$ são coprimos entre si então:

\begin{equation}
    a^{m} \equiv a^{n \mod \varphi(m)} \pmod{m}
\end{equation}
