\subsection{Linear Diophantine Equations}

A Linear Diophantine Equation (in two variables) is an equation of the general form:
 
$$ax + by = c$$ 

Where  $a$ ,  $b$ ,  $c$  are given integers, and  $x$ ,  $y$  are unknown integers.

If $ a = b = 0 $, we have infinite solutions if $ c = 0 $, and $0$ otherwise. 

\subsubsection{Solution(s)}

Let $g = gcd(a,b)$ such that $a x_g + b y_g = g$, then we only have a solution if and only if $g \mid c$, and if it have a solution it have infinite.

The solutions will be of the form :
 
\begin{equation}
  \begin{array}{c}
    x_0 = x_g \cdot \frac{c}{g},
 
    y_0 = y_g \cdot \frac{c}{g}. \\

    a \cdot x_0  + b \cdot y_0 = c 
  \end{array}
\end{equation}

With the initial solution, we can can find every solution, with : 

\begin{equation}
  \begin{array}{c}
    x = x_0 + k \cdot \frac{b}{g}, y = y_0 - k \cdot \frac{a}{g}
  \end{array}
\end{equation}

To find the solution that minimize $x + y$ we use the fact that: 

$$x' = x + k \cdot \frac{b}{g},$$ 
 
$$y' = y - k \cdot \frac{a}{g}.$$ 

Note that  

$x + y$  change as follows:
 
$$x' + y' = x + y + k \cdot \left(\frac{b}{g} - \frac{a}{g}\right) = x + y + k \cdot \frac{b-a}{g}$$ 

If  
$a < b$ , we need to select smallest possible value of  $k$ . If  $a > b$ , we need to select the largest possible value of  $k$ . If  $a = b$ , all solution will have the same sum  $x + y$ .
