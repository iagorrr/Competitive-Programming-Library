\section{Number Theory}

\subsection{Fermat's Theorems and Lemmas}

Let $p$ be a prime number and $a, b \in \mathbb{Z}$:

\begin{equation}
a^p \equiv a \quad (\text{mod } p)
\end{equation}

\begin{equation}
a^{p-1} \equiv 1 \quad (\text{mod } p)
\end{equation}

\begin{equation}
  (a+b)^{p} \equiv a^{p} + b^{p} \quad (\text{mod } p)
\end{equation}

\begin{equation}
  a^{-1} \equiv a^{p-2} \quad (\text{mod } p)
\end{equation}

\begin{equation}
  a^{-1} \equiv a^{p-2} \pmod{p} 
  \quad (\gcd(a, p) = 1)
\end{equation}





% Text about the multiplicative inverse in number theory.

\subsection{Multiplicative Inverse}

\subsubsection{Definition}

A multiplicative inverse of an integer \( a \) modulo \( m \) is an integer \( x \) such that:
\[
a \cdot x \equiv 1 \pmod{m}
\]

\subsubsection{Existence}

A multiplicative inverse exists if and only if \( a \) and \( m \) are coprime, i.e., \( \gcd(a, m) = 1 \).

\subsubsection{Euler totient}
If \( a \) and \( m \) are coprime, the multiplicative inverse
can be computed using Euler's theorem:

\[
x \equiv a^{\varphi(m) - 1} \pmod{m}
\]

If you want to find the smallest $i$ such that $x^i \equiv 1 \pmod{m}$, we can test every divisor of \( \varphi(m) \) until we find the smallest \( i \), such that \( x^i \equiv 1 \pmod{m} \).



\input{content/number_theory/goldbach.tex}

\input{content/number_theory/linear_diophantine.tex}

\input{content/number_theory/wilson.tex}

\input{content/number_theory/chicken_mc_nugget.tex}

\subsection{Critérios de divisibilidade}

\subsubsection{7}

Para verificar a divisibilidade de um número por 7, siga a seguinte regra:

\begin{enumerate}
  \item Pegue o número em questão.
  \item Remova o último dígito (unidade) do número.
  \item Dobre o valor removido no passo anterior.
  \item Subtraia o valor dobrado do número restante.
  \item Se o resultado da subtração for divisível por 7, o número original é divisível por 7.
\end{enumerate}

Exemplo:

Suponha que desejamos verificar a divisibilidade do número 413 por 7.

\begin{enumerate}
  \item Remova o último dígito (3) e dobre-o, obtendo 6.
  \item Subtraia 6 do número restante (41 - 6 = 35).
\end{enumerate}

\subsubsection{11}

\[
n \text{ é divisível por 11} \iff \sum_{i=1}^{k} a_{2i-1} - \sum_{i=1}^{j} a_{2i} \text{ é divisível por 11}
\]

onde \(a_{i}\) é o \(i\)-ésimo dígito do número \(n\), \(k\) é a quantidade de dígitos ímpares, \(j\) é a quantidade de dígitos pares.

Exemplo:

Suponha que desejamos verificar a divisibilidade do número \(n = 7923\) por 11.

\[
k = 2, \quad j = 2
\]

\[
\text{Soma dos dígitos ímpares: } 7 + 3 = 10
\]

\[
\text{Soma dos dígitos pares: } 9 + 2 = 11
\]

\[
\text{Subtração: } 10 - 11 = -1
\]

Como \(-1\) não é divisível por 11, o número \(7923\) não é divisível por 11.

\subsubsection{13}

\begin{equation}
    13 | x \equiv 13 | 4 \cdot (x\%10) + \lfloor x/10 \rfloor
\end{equation}

Em outras palavras 13 divide $x$ se o quádruplo do último algarismo somado com o número sem este algarismo for divisível por 13.

\subsubsection{17}

\begin{equation}
    17 | x \equiv 17 |  \lfloor x/10 \rfloor - 5 \cdot (x \% 10)
\end{equation}

Em outras palavras 17 divide $x$ se o a diferença entre o quíntuplo do último algarismo e o número sem este algarísmo for divisível por 17.

\subsubsection{19}

\begin{equation}
    19 | x \equiv 19 | \lfloor x / 10 \rfloor + 2 \cdot (x \mod 10)
\end{equation}

Em outras palavras 19 divide x se o dobro do último algarismo de x somado a o número restante de x é divisível por 19.

\subsubsection{23}

\begin{equation}
    %23 | x \equiv 23 | \lfloor x / 10 \lflor + 7 \cdot (x \mod 10)  
    23 | x \equiv 23 |  x / 10  + 7 \cdot (x \mod 10)  
\end{equation}



\subsection{Chinese Remainder Theorem}
Let  $m = m_1 \cdot m_2 \cdots m_k$ , where  $m_i$  are pairwise coprime. In addition to  

$m_i$ , we are also given a system of congruences
 
$$
\left\{\begin{array}{rcl}
        x & \equiv & a_1 \pmod{m_1} \\
        x & \equiv & a_2 \pmod{m_2} \\ 
        & \vdots & \\
        x & \equiv & a_k \pmod{m_k}
\end{array}\right.
$$ 

where  $a_i$  are some given constants. The original form of CRT then states that the given system of congruences always has one and exactly one solution modulo  $m$ .

\subsubsection{Corollary: number modulo a non prime number}

A consequence of the CRT is that the equation
 
$$x \equiv a \pmod{m}$$ 

is equivalent to the system of equations
 
$$\left\{\begin{array}{rcl} x & \equiv & a_1 \pmod{m_1} \\ & \vdots & \\ x & \equiv & a_k \pmod{m_k} \end{array}\right.$$ 

(As above, assume that  
$m = m_1 m_2 \cdots m_k$  and  $m_i$  are pairwise coprime).

\subsubsection{Finding solution}

Let $M = m_{1}m_{2} \cdots m_{k}$, and $o_{i} = \frac{M}{m_{i}}$. If the numbers $a_{i}$ satisfy $a_{i}o_{i} \equiv 1 \pmod {m_{i}}$, for every $1 \leq i \leq n$, then a solution for $x$ is $\sum_{i=1}^n a_{i}o_{i}a_{i} \pmod M$



\subsection{Fundamental theorem of arithmetic}

Every integer greater than 1 can be represented uniquely as a product of prime numbers, up to the order of the factors.

\begin{equation}
    n = p_1^{\alpha_1} p_2^{\alpha_2} ... p_k^{\alpha_k}
\end{equation}

\subsubsection{LCM and GCD}

\begin{equation}
\begin{aligned}
    & a = p_1^{\alpha_1} p_2^{\alpha_2} ... p_k^{\alpha_k}  \\
    & b = p_1^{\beta_1} p_2^{\beta_2} ... p_k^{\beta_k} \\
    & (a,b) = p_1^{\min{{\alpha_1, \beta_1}}} p_2^{\min{{\alpha_2, \beta_2}}} ... p_k^{\min{{\alpha_k, \beta_k}}} \\
    & [a,b] = p_1^{\max{{\alpha_1, \beta_1}}} p_2^{\max{{\alpha_2, \beta_2}}} ... p_k^{\max{{\alpha_k, \beta_k}}}
\end{aligned}
\end{equation}


\subsection{Taking modulo at the exponent}

If $gcd(a,m) = 1$ then:

\begin{equation}
    a^{m} \equiv a^{n \mod \varphi(m)} \pmod{m}
\end{equation}
