% Text about the multiplicative inverse in number theory.

\subsection{Multiplicative Inverse}

\subsubsection{Definition}

A multiplicative inverse of an integer \( a \) modulo \( m \) is an integer \( x \) such that:
\[
a \cdot x \equiv 1 \pmod{m}
\]

\subsubsection{Existence}

A multiplicative inverse exists if and only if \( a \) and \( m \) are coprime, i.e., \( \gcd(a, m) = 1 \).

\subsubsection{Euler totient}
If \( a \) and \( m \) are coprime, the multiplicative inverse
can be computed using Euler's theorem:

\[
x \equiv a^{\varphi(m) - 1} \pmod{m}
\]

If you want to find the smallest $i$ such that $x^i \equiv 1 \pmod{m}$, we can test every divisor of \( \varphi(m) \) until we find the smallest \( i \), such that \( x^i \equiv 1 \pmod{m} \).

