\subsection{Chinese Remainder Theorem}
Let  $m = m_1 \cdot m_2 \cdots m_k$ , where  $m_i$  are pairwise coprime. In addition to  

$m_i$ , we are also given a system of congruences
 
$$
\left\{\begin{array}{rcl}
        x & \equiv & a_1 \pmod{m_1} \\
        x & \equiv & a_2 \pmod{m_2} \\ 
        & \vdots & \\
        x & \equiv & a_k \pmod{m_k}
\end{array}\right.
$$ 

where  $a_i$  are some given constants. The original form of CRT then states that the given system of congruences always has one and exactly one solution modulo  $m$ .

\subsubsection{Corollary: number modulo a non prime number}

A consequence of the CRT is that the equation
 
$$x \equiv a \pmod{m}$$ 

is equivalent to the system of equations
 
$$\left\{\begin{array}{rcl} x & \equiv & a_1 \pmod{m_1} \\ & \vdots & \\ x & \equiv & a_k \pmod{m_k} \end{array}\right.$$ 

(As above, assume that  
$m = m_1 m_2 \cdots m_k$  and  $m_i$  are pairwise coprime).

\subsubsection{Finding solution}

Let $M = m_{1}m_{2} \cdots m_{k}$, and $o_{i} = \frac{M}{m_{i}}$. If the numbers $a_{i}$ satisfy $a_{i}o_{i} \equiv 1 \pmod {m_{i}}$, for every $1 \leq i \leq n$, then a solution for $x$ is $\sum_{i=1}^n a_{i}o_{i}a_{i} \pmod M$
