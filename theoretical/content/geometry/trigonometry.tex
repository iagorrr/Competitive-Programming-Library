\subsection{Trigonometry}

\begin{equation}
  \sin(v+w)=\sin v\cos w+\cos v\sin w
\end{equation}

\begin{equation}
  \cos(v+w)=\cos v\cos w-\sin v\sin w
\end{equation}

\begin{equation}
  \tan(v+w)=\dfrac{\tan v+\tan w}{1-\tan v\tan w}
\end{equation}

\begin{equation}
  \sin v+\sin w=2\sin\dfrac{v+w}{2}\cos\dfrac{v-w}{2}
\end{equation}

\begin{equation}
  \cos v+\cos w=2\cos\dfrac{v+w}{2}\cos\dfrac{v-w}{2}
\end{equation}

\begin{equation}
\begin{array}{c}
 (V+W)\tan(v-w)/2{}=(V-W)\tan(v+w)/2 \\
  \text{where $V, W$ are lengths of sides opposite angles $v, w$.}
\end{array}
\end{equation}


\begin{equation}
\begin{array}{c}
	a\cos x+b\sin x=r\cos(x-\phi) \\
	a\sin x+b\cos x=r\sin(x+\phi) \\
  \text{where $r=\sqrt{a^2+b^2}, \phi=\operatorname{atan2}(b,a)$.}
\end{array}
\end{equation}

 
