\section{Bitwise}

\subsection{XOR in range}

Let us introduce the notation $XOR(l,r)=l \oplus (l+1) \oplus \cdots \oplus r$.

Then 

\begin{equation}
  XOR(0, x) = 
  \begin{cases}
    x & \text{if } x \equiv 0 \pmod{4} \\
    1 & \text{if } x \equiv 1 \pmod{4} \\
    x + 1 & \text{if } x \equiv 2 \pmod{4} \\
    0 & \text{if } x \equiv 0 \pmod{4} \\
  \end{cases}
\end{equation}


Then XOR(l,r) can be found as $XOR(0,r) \oplus XOR(0,l−1)$.


\subsection{Binary to gray code}

$G_n = B_n$

$G_{n-1} = B_n \oplus B_{n-1} $

$G_{i} = B_i \oplus B_{i-1} $

$ ... $

$ G_1 = B_2 \oplus B_1 $

\subsection{Gray code to binary}


$ B_n = G_n $

$ B_{n-1} = B_n \oplus G_{n-1} = G_n \oplus G_{n-1} $

$ ... $

$ B_1 = B_2 \oplus G_1 = G_n \oplus G_1 $
