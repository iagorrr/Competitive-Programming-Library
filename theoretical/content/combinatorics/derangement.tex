\subsection{Derangement / Montmort Number / Subfactorial}

In combinatorial mathematics, a derangement is a permutation of the elements of a set in which no element appears in its original position. In other words, a derangement is a permutation that has no fixed points.

The number of derangements of an $n$-element set is called the $n$th derangement number or rencontres number, or the subfactorial of $n$ and is sometimes denoted $!n$ or $D_n$

\begin{equation}
        D_n = !n = n! \sum_{k=0}^{n} \frac{(-1)^k}{k!}
\end{equation}

It also satisfies the following recurrences:

\begin{equation}
        !n = n \cdot !(n - 1) + (-1)^n
\end{equation}

\begin{equation}
        !n = (n - 1)\cdot (!(n - 1) + !(n - 2))
\end{equation}

The first few values, starting from $D_0$: 
\begin{equation}
        \begin{array}{l}
                1, 0, 1, 2, 9, 44, 265, 1854, 14833, 133496, \\
                1334961, 14684570, 176214841, 2290792932, 32071101049,\\
                481066515734, 7697064251745, 130850092279664, 2355301661033953, \\
                44750731559645106, 895014631192902121, 18795307255050944540, \\
                413496759611120779881, 9510425471055777937262
        \end{array}
\end{equation}
